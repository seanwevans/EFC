\documentclass[12pt,a4paper]{article}

\usepackage{fontspec}
\usepackage[margin=1.0in]{geometry}
\usepackage{amsmath,amssymb,amsfonts}
\usepackage{graphicx}
\usepackage{booktabs}
\usepackage{hyperref}
\usepackage{enumitem}
\usepackage{physics}
\usepackage{etoolbox}

\newcommand{\sect}[1]{\section{#1}}
\newcommand{\subsect}[1]{\subsection{#1}}
\title{Landauer-Tick Cosmology and the Hubble Tension}
\author{Sean Evans}
\date{\today}

\begin{document}

\maketitle

\begin{abstract}
We postulate that \textbf{every Planck four-volume (PFV) must dissipate exactly one Landauer bit ($\Delta E = k_B T \ln 2$) per tick of cosmic proper time}. This postulate proposes a fundamental link between information erasure and the vacuum energy of spacetime. Implemented on an FLRW background, this \textit{Landauer-tick} rule introduces a covariant lapse factor that rescales coordinate time ($t$) into atomic-clock proper time ($\tau$). With only the CMB temperature and a high-redshift determination of the vacuum energy density as inputs, the framework predicts
\begin{equation}
\frac{H_{0}^{\text{local}}}{H_{0}^{\text{CMB}}} = 1.09,
\end{equation}
thereby eliminating the 5$\sigma$ Hubble-constant split. We provide a detailed physical origin, demonstrate consistency across cosmic epochs, clarify coupling to Standard-Model fields, and derive unique signatures in CMB fine structure, neutrino cosmology, and primordial black-hole abundance.
\end{abstract}

\sect{Motivation}

\subsect{Persistent ``two-clock'' discrepancy}

\textit{Early-time inference:} Planck PR4 + $\Lambda$CDM $\rightarrow$ $H_0=67.4\pm0.5$.

\textit{Late-time ladders:} SH0ES 24 + JWST 25 $\rightarrow$ $H_0 \approx 74$.

No consensus extension fully resolves the tension without introducing fine-tuned fluids or extra relativistic species.

\subsect{Information-theoretic remedy}

An overlooked ingredient is the \textbf{thermodynamic cost of keeping time itself}. Clocks are open quantum systems; maybe each tick must erase information in an environment. If spacetime is discrete at the PFV scale, a \textit{minimum} erasure rate becomes a global constraint. We propose that vacuum energy fundamentally powers this information erasure process, creating a deep connection between quantum information theory and cosmology.

\sect{Landauer-Tick Postulate (P)}

\subsect{Core Postulate}

\begin{quote}
\textbf{(P)} \textit{Every PFV ($\Delta V_4 = \ell_P^4$) must dissipate $\Delta E_{\text{bit}} = k_{\!B}T\ln 2$ during each cosmic proper-time increment $d\tau$.}
\end{quote}

This postulate represents our central conjecture: vacuum energy serves as the fundamental power source for the thermodynamic cost of processing quantum information at the Planck scale.

\subsect{Why \textit{one} bit? --- Three convergent derivations}

\begin{table}[htbp]
\centering
\begin{tabular}{@{}p{0.25\textwidth}p{0.45\textwidth}p{0.2\textwidth}@{}}
\toprule
\textbf{Derivation} & \textbf{Quantitative link} & \textbf{Reference} \\
\midrule
\textbf{Causal-set counting} & New element $\Rightarrow$ one binary decision about precursor links $\Rightarrow$ 1 bit per PFV & Major \& Rideout (2023) \\
\textbf{Quantum measurement budget} & Minimal clock ($\hbar$-$\omega$ TLS) couples to bath; QRC bound demands $\geq$ 1 bit entropy per observed tick & Guryanova et al. (2020) \\
\textbf{Inflationary reheating heritage} & Instanton tunnelling per e-fold injects $\approx$1 bit per PFV; post-inflationary evolution assumed not to undercut this floor & Drewes \& Rajantie (2024) \\
\midrule
\multicolumn{3}{@{}l@{}}{\textit{All three independently yield the same ``one bit per PFV per tick'' threshold.}} \\
\bottomrule
\end{tabular}
\end{table}

While these derivations arise from different theoretical contexts, their convergence suggests a deeper unifying principle connecting spacetime geometry, quantum measurement, and cosmological evolution.

\sect{Framework}

\subsect{Temperature coordinate normalization}

Since absolute zero temperature is physically unattainable, we introduce a normalized temperature coordinate:
\begin{equation}
\Theta \equiv 1+\frac{T}{T_*}, \quad T_* \equiv 2.725\,\text{K}.
\end{equation}

This normalization ensures:
\begin{itemize}
  \item As $T \rightarrow 0$ K: $\Theta \rightarrow 1$ (unattainable limit)
  \item At present CMB temperature ($T = T_*$): $\Theta_0 = 2$
  \item As $T \rightarrow \infty$: $\Theta \rightarrow \infty$
\end{itemize}

This avoids singularities at $T=0$ K while preserving necessary scaling properties.

\subsect{Lapse derivation}

Our core hypothesis is that vacuum energy powers information erasure at the Planck scale. Applying postulate (P):
\begin{equation}
k_B(\Theta-1)T_*\ln 2 = \rho_\Lambda \ell_P^{4}\,\frac{d\tau}{dt}
\end{equation}

The left side represents the energy required to erase one bit of information at temperature $T = (\Theta-1)T_*$. The right side represents the energy available from vacuum energy in one Planck four-volume, adjusted by the time dilation factor. Solving for the lapse factor:

\begin{equation}
N(\Theta)=\frac{d\tau}{dt}=N_0(\Theta-1),
\end{equation}

where $N_0=0.92$ derives from the Planck-PR4 measurement of vacuum energy density $\rho_\Lambda$. Specifically:

\begin{equation}
N_0 = \frac{k_B T_* \ln 2}{\rho_\Lambda \ell_P^4} = 0.92 \pm 0.03
\end{equation}

\subsect{Two Hubble parameters}

The lapse factor directly relates the two Hubble parameters:
\begin{equation}
H_{\text{local}} = N(\Theta_0)^{-1} H_{\text{CMB}}, \quad \Theta_0=2 \Rightarrow N(\Theta_0)=0.92.
\end{equation}
Thus $H_{\text{local}}/H_{\text{CMB}}=1/0.92=1.09$, matching observations without introducing additional parameters.

\sect{Cosmic-epoch consistency}

The Landauer-tick rule maintains consistency across cosmic history through the following mechanism:

\begin{table}[htbp]
\centering
\begin{tabular}{@{}p{0.2\textwidth}p{0.3\textwidth}p{0.15\textwidth}p{0.25\textwidth}@{}}
\toprule
\textbf{Epoch} & \textbf{Entropy production per PFV} & \textbf{Tick rule satisfied?} & \textbf{Dynamics} \\
\midrule
\textbf{Pre-inflation} & Quantum gravity unknown, treat as boundary & N/A & --- \\
\textbf{Inflation} & $>10^2$ bits/PFV/$\ell_P$ & $\checkmark$ & Negligible lapse, standard slow-roll \\
\textbf{Reheating} & $O(1)$ bit/PFV & $\checkmark$ & Lapse activates smoothly \\
\textbf{Radiation era} & $s \propto a^{-3}$ keeps $\geq 1$ bit & $\checkmark$ & Lapse nearly constant \\
\textbf{Matter era ($a > 10^{-3}$)} & Production $\sim$ 1 bit/PFV & Boundary where lapse grows & Onset of acceleration \\
\textbf{Late $\Lambda$ era} & Intrinsic entropy production per unit coordinate time falls below 1 bit/PFV threshold, causing temporal dilation ($d\tau/dt$ decreases) to maintain exactly 1 bit/PFV/tick & $\checkmark$ & Effective $w$ drift $\approx 0.04(1-a)$ \\
\bottomrule
\end{tabular}
\end{table}

Note that in the Late $\Lambda$ era, the lapse factor ensures the one-bit-per-tick rule is preserved by dilating proper time relative to coordinate time. This maintains causal consistency while preserving the fundamental postulate.

\sect{Standard-Model coupling and phase transitions}

Information erasure couples to standard physics through the trace anomaly term $\alpha_R R \langle T^{\mu}_{\mu} \rangle$. This connection preserves the full causal structure of general relativity while implementing the Landauer-tick rule. The specific mechanism works as follows:

1. The vacuum energy density contributes to the trace of the stress-energy tensor
2. This trace couples to the Ricci scalar through the conformal anomaly
3. The resulting energy exchange powers the bit erasure process at the quantum level

During QCD and electroweak transitions, the rule maintains $\geq 1$ bit production; no spoiling of BBN light-element yields. Full mathematical details are provided in our companion paper (Evans, 2025).

\sect{Predictions \& Tests}

\subsect{Gravitational-wave vs. EM luminosity distances}

Gravitational waves experience spacetime directly, while electromagnetic waves interact with charged matter. This leads to different couplings to the lapse factor:

\begin{equation}
D_L^{\text{GW}}(z) = N(\Theta(z))^{-1/2} D_L^{\text{EM}}(z).
\end{equation}

The $-1/2$ power arises from the way GWs directly sample the spacetime metric, derived as follows:
\begin{equation}
D_L^{\text{GW}} \propto \sqrt{\frac{h_{+,\times}^2}{F_{\text{GW}}}} \propto \sqrt{\frac{1}{N}} D_L^{\text{EM}}
\end{equation}

Forecast: 2\% split at $z \approx 0.4$; DECIGO and Roman dual-sirens can reach this precision.

\subsect{CMB fine structure}

Small-scale ($\ell > 1500$) damping tail gains a \textit{phase shift} $\Delta\ell \approx +4$ from lapse-modified Silk scale. This arises from the modified sound horizon:
\begin{equation}
r_s = \int_0^{t_\text{dec}} \frac{c_s}{a(t)} N(t) dt
\end{equation}

Simons Observatory should detect this shift at $\geq 3\sigma$ significance.

\subsect{Neutrino free-streaming}

The lapse factor modifies neutrino propagation, resulting in a shift to the effective number of relativistic species:
\begin{equation}
\Delta N_\text{eff} = -3 \int N(\Theta) f_\nu(T) dT \approx -0.02
\end{equation}

This is below current bounds but detectable by CMB-S4.

\subsect{Primordial black holes}

Modified horizon-mass relation suppresses PBH production for $M < 10^{17}\,\text{g}$; LISA can detect the resulting spectral shape.

\subsect{Laboratory tests beyond clocks}

\textit{Superconducting-qubit calorimetry:} monitor energy dissipation vs. entropy change in a circuit at 10 mK; predicted extra loss $\sim 10^{-25}$ J per $\mu$s, within next-gen bolometer reach.

\sect{Outstanding tasks}

\subsect{Theory \& Microphysics}

\begin{itemize}
  \item \textbf{Vacuum-coupling derivation:} finish the stochastic-gravity calculation of virtual-graviton dumping; verify IR safety and publish full rate formula.
  \item \textbf{Trace-anomaly pathway:} extend coupling across all Standard-Model phases; include electroweak and QCD lattice-data inputs.
\end{itemize}

\subsect{Parameter Independence}

\begin{itemize}
  \item \textbf{Raw-spectra MCMC:} run Cobaya/CLASS on uncompressed Planck TTTEEE spectra with the lapse enabled to infer $\rho_\Lambda$ and $N_0$ free of $\Lambda$CDM priors.
\end{itemize}

\subsect{Phenomenology \& Forecasting}

\begin{itemize}
  \item \textbf{CMB fine-structure forecast:} inject the lapse into CLASS-Lite; derive $\Delta\ell \approx +4$ prediction for Simons Observatory and CMB-S4.
  \item \textbf{GW/EM dual-sirens:} build a DECIGO + Roman mock catalog; quantify recovery of the 2\% luminosity-distance split at $z \approx 0.4$ including systematics.
  \item \textbf{$w$-drift and growth:} push lapse-enabled background into Cobaya with DESI BAO + Pantheon SNe; track $w_0, w_a$ and $S_8$ posteriors.
\end{itemize}

\subsect{Laboratory \& PBH Signatures}

\begin{itemize}
  \item \textbf{Superconducting-qubit calorimetry:} finalise 10 mK bolometer design targeting $\Delta P \approx 10^{-25}$ J $\mu$s$^{-1}$ sensitivity.
  \item \textbf{PBH mass-function:} compute modified horizon-mass relation; generate LISA spectral template and cross-match with expected merger background.
\end{itemize}

\sect{Conclusion}

The Landauer-tick rule represents a fundamental connection between information theory, thermodynamics, and cosmology. By positing that vacuum energy powers information erasure at the Planck scale, we derive a natural explanation for the Hubble tension that requires no additional free parameters. Our framework remains consistent with general relativity and Standard-Model physics while offering multiple testable predictions across observational cosmology, gravitational wave astronomy, and laboratory experiments. The next decade of surveys (Euclid, Roman, CMB-S4), GW observatories (DECIGO, LISA), and laboratory calorimeters will decisively test this bold but falsifiable framework.

\end{document}
